\section{Desenvolvimento}

Durante o desenvolvimento, foi elaborado o repositório chamado ``filipecancio/sbc-template'', desenvolvido utilizando integração contínua, implementando primeiramente pipelines de verificação do LaTeX a cada salvamento de arquivo ``.tex'', pipelines para escolha de um arquivo ``.pdf'' alvo e, posteriormente, uma verificação final a cada commit gerado e submetido ao site do GitHub. Em paralelo, foi utilizada a entrega contínua para a geração automática de ``.pdf'' e seus formatos de disponibilização. Ao total, foram três formatos: geração de ``.pdf'' instantânea a cada arquivo salvo, geração para versão de release e geração de ``.pdf'' a cada versão de Pull Request.

Para facilitar o uso do projeto para usuários comuns, toda a parte escrita é restrita à pasta ``article''. Dentro dela, temos um arquivo principal chamado ``main.tex'', uma pasta chamada ``sections'', onde há arquivos ``.tex'' de exemplo que podem ser substituídos ou reescritos, uma pasta de imagens e uma pasta chamada ``util'', usada para configuração interna do LaTeX. 
Foi elaborado um manual de uso, presente no arquivo ``README.md'' do projeto, com um passo a passo completo.

Toda a implementação CI/CD (integração contínua e entrega contínua) foi realizada na elaboração do Devcontainer do projeto, presente na pasta ``.devcontainer'' (Imagem~\ref{fig:image12}). Nos próximos tópicos, veremos em detalhes a elaboração do ambiente de desenvolvimento e das automações presentes.


\begin{figure}[ht]
	\centering
	\includegraphics[width=.5\textwidth]{./images/image12.png}
	\caption{Estrutura de arquivos do devcontainer}
	\label{fig:image12}
\end{figure}