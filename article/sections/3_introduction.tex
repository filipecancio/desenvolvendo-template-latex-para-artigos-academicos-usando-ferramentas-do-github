\section{Introdução}

A sociedade acadêmica utiliza as mais variadas plataformas para a elaboração de textos acadêmicos, como o Word, Google Docs, e Overleaf. O uso de funcionalidades como compatilhamento online em tempo real do Google Docs e histórico de versão, permite aos pesquisadores uma facilidade na hora de tornar seus projetos em um documento público, principalmente se for um trabalho conjunto, seja entre autor e orientador, ou entre autores.
Nesse mercado de suítes de escritório, o Overleaf se destaca por ter uma interface amigável assim como os demais concorrentes, porém fazendo o uso da linguagem de marcação LaTex, que possibilita a elaboração do conteúdo científico sem os atrasos de configuração, normas e formatação de texto.
Apesar de todas as facilidades que essas ferramentas fornecem, existem limitações de mercado, que obrigam ao usuário a pagar por um plano que permita o uso, como é o caso do Word e do Overleaf, ou que possuam limitações no controle de versão como é o caso do Google Docs e do Word.
O artigo presente aborda uma elaboração de uma alternativa a essa plataformas de desenvolvimento de gêneros acadêmicos. Inspirado na ferramenta overleaf, o projeto utiliza da tecnologia LaTex para criar um ambiente de desenvolvimento de artigos capaz de gerar arquivos no formato .pdf em tempo real e diferente momentos de controle de versão, com o mínimo de conhecimento de LaTex necessário de forma totalmente gratuita, seja online ou salvo no computador.