\subsection{Actions}

O GitHub Actions é uma plataforma criada pelo GitHub que utiliza de containers docker para a realização de integração contínua e entrega cotínua (CI/CD)~\cite{github:02}. Através do GitHub Actions podemos criar testes automatizados, validações de projeto, criar versões compiladas de arquivos, entre outras atividades~\cite{github:02}. Em nosso projeto utilizamos o GitHub Actions para duas atividades, em ambos os casos para gerar um pdf final do projeto. O que diferencia em cada caso é que temos na pasta .github (figura~\ref{fig:image15}), o arquivo $latex\_build.yml$ que gera um PDF em branchs de pull requests abertas, ao subir um novo commit, o pdf é gerado e em seguida é criado um comentário na Pull Request. Já o arquivo $latex\_release.yml$ configura o PDF para virar uma versão de release na página inicial do repositório.

\begin{figure}[ht]
	\centering
	\includegraphics[width=.5\textwidth]{./images/image15.png}
	\caption{Arquivos $latex\_build.yml$ e $latex\_release.yml$ respectivamente}
	\label{fig:image15}
\end{figure}

A vantagem do GitHub actions é permitir a edição e atualização do projeto sem necessáriamente criar um ambiente devcontainer para edições rápidas e discussões entre autor e orientador. Ao alterar qualquer arquivo no site do GitHub, as actions se encarregam de gerar um novo artigo final em PDF.