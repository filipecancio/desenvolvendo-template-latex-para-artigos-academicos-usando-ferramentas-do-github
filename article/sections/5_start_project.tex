\section{Utilização do projeto}

A solução apresentada consiste em um repositório template do Github  que se converte em um ambiente de desenvolvimento que podemos chamar de Devcontainer. Apesar do nome, não  há necessidade alguma de conhecimento em programação e possui as seguintes funcionalidades:

\begin{itemize}
	\item Colaborativo: Utiliza 100\% de ferramentas do github para escrever artigos com orientadores e outros autores.
	\item Simples mas completo: Basta clonar e editar os textos. Com um conhecimento básico de LaTex que o próprio repositório fornecem nas instruções iniciais já é possível escrever um artigo completo, mas é util também para pessoas com o conhecimento avançado em LaTex.
	\item Online e Offline: Você pode utilizar o navegador para editar, mas é possivel acessar de forma offline também.
\end{itemize}

As edições podem ocorrer de três formas: diretamente no github pelo repositório, de forma online pelo github (figura~\ref{fig:fig01}), pela plataforma codespaces (figura~\ref{fig:fig02}), ou de forma offline pelo Visual Studio Code (figura~\ref{fig:fig03}).


\begin{figure}[ht]
	\centering
	\includegraphics[width=.6\textwidth]{./images/fig01.png}
	\caption{Utilizando o projeto no site do GitHub}
	\label{fig:fig01}
\end{figure}

\begin{figure}[ht]
	\centering
	\includegraphics[width=.6\textwidth]{./images/fig02.png}
	\caption{Utilizando o projeto pelo Codespaces}
	\label{fig:fig02}
\end{figure}

\begin{figure}[ht]
	\centering
	\includegraphics[width=.6\textwidth]{./images/fig03.png}
	\caption{Utiizando o projeto com Visual Studio Code}
	\label{fig:fig03}
\end{figure}