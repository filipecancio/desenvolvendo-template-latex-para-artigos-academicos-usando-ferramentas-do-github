\section{Utilizando o projeto}

Após a criação de um novo projeto podemos usar de duas formas: através de uma plataforma online com todas a ferramentas configuradas (codespaces). Ou configurando o projeto na máquina com o visual studio code.


\subsection{Utilizando de forma online com o Github Codespaces}

Para acessar o codespaces basta acessar a opção ``code > codespaces > create codespaces on main''. como na figura~\ref{fig:image04}

\begin{figure}[ht]
	\centering
	\includegraphics[width=.5\textwidth]{./images/image04.png}
	\caption{Criando um projeto novo no codespaces}
	\label{fig:image04}
\end{figure}

A partir daqui será redirecionada uma nova tela com uma versão online do Visual Studio Code com todo o projeto já configurado, pronto para ser usado.

\subsection{Utilizando de forma offline com o Visual Studio Code}

Para utilizar a versão offline é necessário ter instalado o git, MiKTeX o visual studio code e suas extensões (
LaTeX Workshop, latex-formatter, LaTeX LTeX,code runner). Após instalados basta fazer o clone do projeto e abrir no visual studio code.


A utilização do codespaces permite edição do projeto em qualquer dispositivo com internet, independente de suas especificações, porém usar o projeto na versão offline permite mais estabilidade no uso, visto que não é necessário o uso de internet.
Ao abrir o editor de texto escolhido haverá uma série de arquivos dispostos nas seguintes pastas:

- devcontainer: configuração padrão do projeto para rodar o LaTex de forma automática
- github: configuração de ações para criação de pdfs a cada versão do projeto.
- article: pasta contendo os arquivos Latex. Aqui faremos a edição do nosso texto adêmico.

Dentro da pasta ``article'' podemos encontrar:
