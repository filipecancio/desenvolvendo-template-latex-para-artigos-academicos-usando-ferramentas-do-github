\section{Sobre o projeto}

A solução apresentada foi desenvolvida em um repositório do Github chamado ``filipecancio/sbc-template'' (figura~\ref{fig:fig10}) possuindo o template LaTex oficial da SBC (Sociedade Brasileira de Computação) para a produção de artigos. O acesso está disponível em ``https://github.com/filipecancio/artigo-final''. O repositório, ao ser acessado, se converte em um ambiente de desenvolvimento (Devcontainer), aonde não há necessidade de conhecimento em programação, basta editar qualquer arquivo ``.tex'' e é gerado um ``.pdf'' formatado no padrão SBC.

\begin{figure}[ht]
	\centering
	\includegraphics[width=.7\textwidth]{./images/fig10.png}
	\caption{Repositório filipe/sbc-template}
	\label{fig:fig10}
\end{figure}

\subsection{Funcionalidades}

Como solução alternativa às plataformas apresentadas anteriormente, o projeto possui algumas funcionalidades semelhantes porém, mais abrangentes e disponíveis de forma gratuita. Podemos visualizar abaixo um breve comparativo entre as plataformas.

\begin{table}[ht]
	\centering
	\begin{tabular}{|c|c|c|c|c|}
		\hline
		Plataformas & Usa LaTex & Colaborativo & Controle de versão & Offline
		\\
		\hline
		Overleaf & Sim & Só na versao paga & Só na versao paga & Não \\
		\hline
		Microsoft Word & Não & Sim & Limitado & Sim \\
		\hline
		Google Documentos & Não & Sim & Limitado & Não \\
		\hline
		filipecancio/sbc-template & Sim & Sim & Sim & Sim \\
		\hline
	\end{tabular}
	\caption{Comparativos entre plataformas}
	\label{tab:tabela01}
\end{table}

Dentre as funcionalidades citadas acima (tabela~\ref{tab:tabela01}), podemos destacar as seguintes características do projeto ``filipecancio/sbc-template'':

\begin{itemize}
	\item Colaborativo: As ferramentas Word e Google Documentos utilizam serviços de nuvem (OneDrive e Drive, respectivamente) para compartilhar documentos. O Overleaf necessita de um plano pago para executar esse compartilhamento entre diferentes contas. Já o projeto ``filipecancio/sbc-template'' utiliza as ferramentas do GitHub, que são gratuitas, para realizar o compartilhamento.
	\item Controle de versão: O Word e o Google Documentos possuem um controle de versão que permite apenas visualizar histórico de edições recentes de diferentes usuários. O Overleaf permite integração com GitHub e GitLab para controle de versão, porém precisa de um plano pago para habilitar a integração. O projeto ``filipecancio/sbc-template'' utiliza o controle de versão do Git (integrado ao GitHub), que possui toda uma estrutura para o desenvolvimento de software e essa estrutura é direcionada para o versionamento do artigo.
	\item Offline: O Word possui versão offline, porém necessita de um plano pago. O Overleaf e o Google Documentos não possuem versão offline. O projeto ``filipecancio/sbc-template'' utiliza o Visual Studio Code para edição offline.
\end{itemize}

O projeto possui, além das caracteristícas acima, uma versatilidade para diferentes perfis de usuário. Mesmo voltado para iniciantes em LaTex e pessoas que não estão acostumadas com linguagens de programação, é útil para pessoas com conhecimento avançado nessa linguagem de marcação. 

O usuário deve possuir uma conta no GitHub para acessar o repositório ``filipecancio/sbc-template'', clonar com o nome desejado, de preferência com o nome do artigo, e começar a editar os arquivos de acordo ao manual de instruções disponível no arquivo ``README.md''.

As edições podem ocorrer de três formas:


\begin{itemize}
	\item Pelo  site do GitHub: Acessando pelo, você pode clicar diretamente nas pastas do diretório e nos arquivos correspondentes. O site permite a edição e dependendo da remificação escolhida, gera um ``.pdf'' após a edição(figura~\ref{fig:fig01}).
	
	\begin{figure}[ht]
		\centering
		\includegraphics[width=.6\textwidth]{./images/fig03.png}
		\caption{Utilizando o projeto no site do GitHub}
		\label{fig:fig01}
	\end{figure}

	\item Pela plataforma Codespaces: O GitHub disponibiliza a edição  de forma mais completa utilizando o Codespaces, uma ferramenta que simula uma máquina e o Visual Code Studio no navegador, que pode ser acessada seguindo as instruções do repositório. O acesso pelo Codespaces permite o uso de todas funcionalidades de forma online (figura~\ref{fig:fig02}).
	
	\begin{figure}[ht]
		\centering
		\includegraphics[width=.6\textwidth]{./images/fig02.png}
		\caption{Utilizando o projeto pelo Codespaces}
		\label{fig:fig02}
	\end{figure}

	\item Pelo Visual Studio Code: para o acesso offline, basta clonar o projeto seguindo as instruções no repositório e acessar o Visual Studio Code (figura~\ref{fig:fig03}).
	
	\begin{figure}[ht]
		\centering
		\includegraphics[width=.6\textwidth]{./images/fig01.png}
		\caption{Utilizando o projeto com Visual Studio Code}
		\label{fig:fig03}
	\end{figure}
\end{itemize}

\clearpage